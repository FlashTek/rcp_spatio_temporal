\section{Fazit}
Wie die Ergebnisse aus Abschnitt \ref{chp:applications} zeigen, bieten \textsc{Echo State Networks} eine gute Möglichkeit zur Vorhersage und Kreuz-Vorhersage von zeitlichen Messreihen. Hierbei konnten teilweise sogar in Abschnitt \ref{sec:mackey_glass} die bisherigen Ergebnisse, welche mit anderen Methoden erzielt worden sind, übertroffen werden. Es hat sich gezeigt, dass für die meisten Anwendungen bereits eine kleine Reservoirgröße $N \leq 2000$ ausreichend ist. Dies hat zur Folge, dass sowohl der Trainings als auch der Testvorgang sehr schnell und ressourcenschonend durchgeführt werden konnten.\\

Ein Nachteil ist allerdings, dass die Modelle über viele Hyperparameter verfügen, welche alle anwendungsspezifisch angepasst werden müssen. In den meisten Fällen konnten diese allerdings relativ schnell so eingestellt werden, dass zufriedenstellende Ergebnisse erzielt werden konnten. Hierbei hat es sich als sehr hilfreich erwiesen erst grobe Parameterbereiche abzutasten, um eine Vorauswahl von vielversprechenden Hyperparametern zu finden, bevor diese fein abgetastet werden.\\
Für weitere Arbeiten könnte sich das Implementieren eines Algorithmus, der diese Parameter selbstständig schnell optimiert als hilfreich erweisen. Eine solche Methode ist in \citep{jaeger2007} vorgestellt worden, doch ist sie im Rahmen dieses Praktikums auf Grund der angegebenen Instabilität des Algorithmus nicht implementiert worden.