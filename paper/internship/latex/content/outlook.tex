\chapter{Ausblick}
Für das Schreiben des Berichtes für dieses Spezialisierungspraktikums hat sich die Wahl der \textsc{LaTeX}-Umgebung als gut geeignet herausgestellt. Die optisch Darstellung von mathematischen Ausdrücken sowie des erklärenden Textes ist hierbei intuitiver umzusetzen als bei den bekannten Alternativen.\\

Das Programmieren der Anwendungsbeispiele wurde mit der Sprache \textsc{Python} und den bekannten Frameworks \textsc{Numpy} und \textsc{Scipy} umgesetzt. Diese Entscheidung wurde hauptsächlich durch die große Flexibilität dieser Sprache und der ausreichenden Leistung der Bibliotheken beeinflusst. Die hierbei entstandenen Grafiken wurden mit der \textsc{Pyplot} Bibliothek umgesetzt.\\ 

Während des Praktikums wurde für die Literaturrecherche der Dienst \textsc{Google Scholar} benutzt. Dies liegt daran, dass er nicht nur eine umfassende Datenbank besitzt, sondern auch, dass das Erhalten der indizierten Werke sehr einfach ist.\\
Die benutzte Literatur wurde über \textsc{BibTeX} direkt in den \textsc{LaTeX}-Quellcode des Berichts eingebunden und zitiert.\\

Dieses gesamte Vorgehen hat sich im Laufe des Praktikums als gut erwiesen und wird somit auch in der anschließenden Bachelorarbeit weiterverfolgt werden.