\subsection{Überblick}
Um die (Leistungs)Probleme der \textsc{RNN} zu umgehen, wurden als mögliche Lösung die \textsc{Echo State Networks} von H. Jäger vorgeschlagen \cite{jaeger2010}. Etwa zeitgleich wurde von W. Maas das Modell der \textit{Liquid State Machines} vorgeschlagen. In diesem Modell steht der biologische Hintergrund im Fokus, doch sind die Ergebnisse denen der \textsc{Echo State Networks} sehr ähnlich \citep{Maass2011}. 

\subsection{Aufbau}
Ein \textsc{ESN} ist eine Spezialform eines \textsc{RNN}s. Hierbei wird eine auf dem ersten Blick eigenartige Entscheidung getroffen: Während des gesamten Trainingsvorganges werden die Verbindungen der einzelnen Einheiten größtenteils nicht verändert. Es wird versucht durch das \textit{Echo} der vorherigen Signale, welche noch im Netzwerk gespeichert sind, diese Signale zu rekonstruieren - hier raus ergibt sich auch der Name \cite{lukoseviciusa2009}. Im Folgenden wird der Aufbau und anschließend die Funktionsweise eines solchen Netzwerkes beschrieben.\\

Allgemein bildet das Netzwerk $S$ ein zeitliches Signal $\vec{u}(n) \in \mathbb{R}^{N_u}$  auf eine zeitlich variable Ausgabe $\vec{y}(n) \in \mathbb{R}^{N_y}$ für die Zeiten $n=1, ..., T$ ab. Zudem besitzt das System ein sogenanntes \textsc{Reservoir} aus $N$ nicht-linearen Einheiten. Der innere Zustand des Netzwerkes wird durch diese Einheiten beschrieben und als $s(n) \in \mathbb{R}^{N}$ bezeichnet.\\
Die Verbindungen der inneren Einheiten untereinander werden durch die Gewichtsmatrix $\mathbf{W} \in \mathbb{R}^{N \times N}$ beschrieben. Das Eingangssignal wird zusammen mit einem \textit{Bias} $b_{in} \in \mathbb{R}$ durch die Matrix $\mathbf{W}_{in} \in \mathbb{R}^{N \times (N_u+1)}$ auf die inneren Einheiten weitergeleitet.

Die zeitliche Entwicklung der inneren Zustände berechnet sich nach der Vorschrift
\begin{align}
\label{eq:esn_stateeq}
\vec{s}(n) = (1 - \alpha) \cdot \vec{x}(n-1)  \alpha \cdot f_{in}\left( \mathbf{W}_{in} [b_{in}; \vec{u}(n)] + \mathbf{W} \vec{x}(n-1) \right),
\end{align}
wobei $f_{in}$ eine beliebige (meistens \textit{sigmoid}-förmige) Transferfunktion ist, und $[\cdot\,;\,\cdot]$ das vertikale Aneinanderfügen von Vektoren beziehungsweise Matrizen bezeichnet. Für diese Zustandsgleichung wurde das Modell eines \textit{Leaky Integrator Neurons} genutzt, wobei $\alpha \in (0,1]$ die Verlustrate beschreibt. Für $\alpha=1$ ergibt sich als Spezialfall ein gewöhnliches Neuron
\begin{align}
\vec{s}(n) = f_{in}\left( \mathbf{W}_{in} [b_{in}; \vec{u}(n)] + \mathbf{W} \vec{x}(n-1) \right).
\end{align}

Da für manche Anwendungsfälle auch eine direkte Rückkopplung wünschenswert ist, kann das System noch um eine Ausgabe-Rückkopplung erweitert werden. Diese verbindet die Ausgabe erneut mit den inneren Einheiten durch die Matrix $\mathbf{W}_{fb} \in \mathbb{R}^{N \times N_y}$.
Somit ergibt sich 
\begin{align}
\label{eq:esn_stateeq_feedback}
\vec{s}(n) = (1 - \alpha) \cdot \vec{x}(n-1)  \alpha \cdot f_{in}\left( \mathbf{W}_{in} [b_{in}; \vec{u}(n)] + \mathbf{W} \vec{x}(n-1) + \mathbf{W}_{fb} \vec{y}(n) \right)
\end{align}
als Zustandsgleichung.\\

An Hand der inneren Zustände lassen sich nun noch die sogenannten erweiterten inneren Zustände $x(n) = [b_{out}; \vec{s}(n); \vec{u}(n)] \in \mathbb{R}^{1 + N_u + N}$ definieren, wobei $b_{out}$ ein \textit{Bias} für die Ausgabe darstellt. 

Aus diesen erweiterten inneren Zuständen kann nun die Ausgabe $\vec{y}(n)$ konstruiert werden. Dies kann entweder im Sinne einer Linearkombination durch die Ausgangsmatrix $\mathbf{W}_{out} \in \mathbb{R}^{(1 + N_u + N) \times N_y}$ oder durch andere nicht lineare Klassifizierer wie beispielsweise einer \textsc{Support Vector Machine (SVM)} durchgeführt werden. Im Folgenden wird nur der Fall einer Linearkombination betrachtet, da sich für die anderen Methoden ein analoges Verfahren ergibt.
In diesem Fall berechnet sich die Ausgabe mittels
\begin{align}
\vec{y}(n) = f_{out} \left( \mathbf{W}_{out} \vec{x}(n) = \mathbf{W}_{out} [b_{out}; \vec{s}(n); \vec{u}(n)] \right),
\end{align}
wobei $f_{out}$ die Transferfunktion der Ausgabe ist.\\

Während die Matrix $\mathbf{W}_{out}$ durch den Trainingsvorgang bestimmt wird, werden die Matrizen $\mathbf{W}_{in}$ und $\mathbf{W}$ a priori generiert und festgelegt. Hierbei hat sich für das Generieren der Eingangsmatrix eine zufällige Anordnung von zufälligen Gleitkommazahlen zwischen $-1.0$ und $1.0$ als geschickt herausgestellt. Falls ein Feedback gewünscht ist, also Gleichung (\ref{eq:esn_stateeq_feedback}) genutzt wird, wird $\mathbf{W}_{fb}$ gleichartig konstruiert. Auf das Generieren der inneren Matrix $\mathbf{W}$ wird in Abschnitt \ref{sc:theory} genauer eingegangen.

\subsection{Trainingsvorgang}
Nachdem der Aufbau des Netzwerkes beschrieben ist, ergibt sich nun die Frage, wie der Trainingsvorgang durchführt wird.

Hierfür wird für die Zeiten $n=0, ..., T_0$ das \textsc{ESN} mit dem Signal $\vec{u}(n)$ betrieben, wobei $T_0$ die \textit{transiente Zeit} beschreibt. Hierdurch soll das System aus seinem zufällig gewähltem Anfangszustand in einen charakteristischen Zustand übergehen. Anschließend wird das System für Zeiten $n < T$ weiter betrieben und die erweiterten Zustände $\vec{x}(n)$ als Spalten in der \textit{Zustandsmatrix} $\mathbf{X} \in \mathbb{R}^{(1 + N_u + N) \times T}$ gesammelt. Analog dazu werden die gewünschten Ausgaben $\vec{y}(n)$ nach dem Anwenden der Inversen $f^{-1}_{out}$ der Ausgabe-Transferfunktion $f_{out}$ auch als Spalten in der \textit{Ausgabematrix} $Y \in \mathbb{R}^{N_y \times T}$ gesammelt.
Nun wird eine Lösung der Gleichung
\begin{align}
\mathbf{Y} = \mathbf{W}_{out} \mathbf{X}
\end{align}
für $\mathbf{W}_{out}$ gesucht. Hierfür stehen mehrere Verfahren zur Verfügung, von denen zwei prominente erwähnt sein sollen.
Zum einen kann die Lösung durch eine \textit{Tikhonov Regularisierung} mittels der Regularisierung $\beta \cdot ||{\vec{W}_i}||^2$ mit der Konstante $\beta$ erhalten werden. Hierbei steht $\vec{W}_i$ für die jeweils $i$-te Zeile der Gewichtsmatrix. Das Verfahren
\begin{align}
\label{eq:tikhonov}
\mathbf{W}_{out} = \mathbf{Y} \mathbf{X}^T \left(\mathbf{X} \mathbf{X}^T + \beta I \right)^{-1}
\end{align}
ist sehr leistungsstark, aber auch teilweise numerisch instabil. Es kann allerdings bei geeigneter Wahl von $\beta$ die besten Ergebnisse erzielen \cite{lukoseviciusa2009}.\\

Zum anderen kann zur Lösung die \textit{Moore-Penrose-Pseudoinverse} $\mathbf{X}'$ genutzt werden, sodass für die Ausgabematrix
\begin{align}
\label{eq:pseudo_inverse}
\mathbf{W}_{out} = \mathbf{Y} \mathbf{X}'
\end{align}
folgt. Dieses Verfahren ist zwar sehr rechenaufwendig aber dafür numerisch stabil \cite{lukoseviciusa2009, jaeger2012}. Nichts desto trotz, kann allerdings auf Grund des Fehlens einer Regularisierung leicht der Effekt des \textsc{Overfittings} auftreten. Dieser kann umgangen werden, indem in der Zustandsgleichung (\ref{eq:esn_stateeq}) beziehungsweise (\ref{eq:esn_stateeq_feedback}) eine leichte normalverteilte Störung $\vec{\nu}(n)$ der Größenordnung $\num{1e-1}$ bis $\num{1e-5}$ addiert wird. Dieser Ansatz beruht auf Empirie, da eine mathematische Begründung hierfür noch nicht vollständig gelungen ist \citep{jaeger2010, lukoseviciusa2009}.\\

Zusammenfassend ergibt sich somit der folgende Funktionsablauf für die Anwendung eines \textsc{ESN}:

\singlespacing
\begin{enumerate}
	\item Zufälliges Generieren der Matrizen $\mathbf{W}_{in}, \mathbf{W}_{fb}$ und Konstruktion der Matrix $\mathbf{W}$ 
	\item Einspeisen des Signals $u(n)$ und Konstruktion der Zustandsmatrix $\mathbf{X}$ und der Ausgabematrix $\mathbf{Y}$ 
	\item Berechnung der Ausgabematrix $\mathbf{W}_{out}$
	\item Einspeisen des Signals $u(n)$ für Vorhersagen des Signales $y(n)$ für $n > T$
\end{enumerate}
\onehalfspacing