\section{Einleitung}
Mittels \textit{Neuronaler Netzwerke} konnten in den vergangenen Jahren viele Probleme des \textit{Machine Learnings} und der datenbasierten Ereignisvorhersage gelöst werden. Hierbei hat sich allerdings die Vorhersage von Zeitserien lange Zeit als problematisch erwiesen - dies änderte sich erst, als rekurrente Netzwerke vermehrt genutzt wurden. Eine Form dieser Netzwerke sind die \textsc{Echo State Networks} aus dem Bereich des \textit{Reservoir Computings}. Sie stellen einen vereinfachten Ansatz dar, welche teilweise bemerkenswerte Ergebnisse bei der Analyse und Vorhersage von Zeitserien liefern kann.\\  

Dieser Bericht gibt eine Übersicht über die im Spezialisierungspraktikum gewonnen Erkenntnisse bezüglich \textsc{Echo State Networks}. Hierbei wurden zuerst die theoretischen Grundlagen betrachtet und die dabei gewonnenen Informationen anschließend auf zwei Anwendungsbeispiele bezogen.\\

Alle Programmierbeispiele während des Praktikums wurden in \textsc{Python} mittels \textsc{numpy} und \textsc{scipy} erstellt. Die relevanten Auszüge hiervon sind im Anhang zu finden.