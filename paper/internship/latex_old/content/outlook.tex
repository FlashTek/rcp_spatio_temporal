\section{Ausblick}
Für das Schreiben dieses Berichts hat sich die Wahl der \textsc{LaTeX}-Umgebung als gut geeignet herausgestellt. Die optische Darstellung von mathematischen Ausdrücken sowie des erklärenden Textes ist hierbei intuitiver umzusetzen als bei den bekannten Alternativen. Hierbei wurde die verwendete Literatur über \textsc{Zotero} und \textsc{BibTeX} direkt in den \textsc{LaTeX}-Quellcode eingebunden und zitiert.\\
Für die Literaturrecherche ist der Dienst \textsc{Google Scholar} benutzt worden, welcher eine umfassende Datenbank besitzt.\\

Das Programmieren der Anwendungsbeispiele wurde mit der Sprache \textsc{Python} und den bekannten Frameworks \textsc{Numpy} und \textsc{Scipy} umgesetzt. Diese Entscheidung wurde hauptsächlich durch die große Flexibilität dieser Sprache und der ausreichenden Leistung der Bibliotheken beeinflusst. Die hierbei entstandenen Grafiken wurden mit der \textsc{Pyplot} Bibliothek umgesetzt.\\ 

Dieses gesamte Vorgehen hat sich im Laufe des Praktikums als gut erwiesen und wird somit auch in der anschließenden Bachelorarbeit weiterverfolgt werden.