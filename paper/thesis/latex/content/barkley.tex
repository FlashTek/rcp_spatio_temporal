\section{Barkley Modell}
Das \textit{Barkley}-Modell, welches 1990 von Dwight Barkley vorgestellt wurde, ist ein System aus gekoppelten Reaktionsdiffusionsgleichungen. Dies sind partielle Differentialgleichungen (\textit{PDE}) zweiter Ordnung welche einen Diffusionsterm besitzen. Das \textit{Barkley}-Modell beschreibt ein erregbares und oszillierendes Medium. Das Modell besteht aus zwei Variablen $u(t)$, $v(t)$ die den  \textit{PDEs}
\begin{equation}
\begin{gathered}
\frac{\partial u}{\partial t} = D \cdot \nabla^2 u + \frac{1}{\epsilon} (1-u) \left(u-\frac{v+b}{a}\right)\\
\frac{\partial v}{\partial t} = u^\alpha-v,
\end{gathered}
\end{equation}
unterliegen. Dabei ermöglicht $\alpha=1$ dem System \textit{periodische} Wellenmuster auszubilden und $\alpha=3$ bedingt ein \textit{chaotisches} Verhalten. Im Folgenden wir stets der Fall $\alpha=3$ betrachtet. Die Variable $u(t)$ durchläuft hierbei eine schnellere Dynamik als die hemmende Variable $v(t)$ \citep{Barkley:2008, berg2011synchronization}. Das Modell kann genutzt werden um die Dynamik der Herzgewebes zu beschreiben. Dabei nimmt die Variable $u$ die Rolle einer Membranspannung ein.\\
Die Parameter $\epsilon, b$ und $a$ charakterisieren das Verhalten des Systems und werden in der gesamten folgenden Arbeit - sofern keine anderen Angaben existieren - als
\begin{align*}
a &= 0.8,\\ b &= 0.01,\\ \epsilon &= 0.02
\end{align*}
festgelegt.\\
Zudem wird das Modell in dieser Arbeit in zwei Dimensionen betrachtet, sodass $u(t, x,y)$ und $v(t, x,y)$ skalare zeitabhängige Felder sind.\\
Zur Simulation des System werden zunächst die \textit{PDE}s zeitlich mit einem Zeitschritt $\Delta t$ und örtlich mit einer Gitterkonstante $\Delta x$ diskretisiert. Zur Beschreibung des Diffusionstermes wird eine fünf-Punkte Methode
\begin{align}
\nabla^2 u(t)_{i,j} \simeq \frac{u(t)_{i-1, j} + u(t)_{i+1,j} + u(t)_{i,j-1} + u(t)_{i,j+1} - 4 u(t)_{i,j}}{\Delta x^2} \eqdef \Sigma(t)_{i,j}
\end{align} 
nach \citep{Barkley:2008} verwendet. Dabei stehen die tiefergestellten Indizes für den diskretisierten Ort der \textit{x-y-Ebene}.
Für kleine Zeitschritte $\Delta t$ ist ein \textit{explizites Eulerverfahren}
\begin{equation}
\begin{gathered}
u(t+1)_{i,j} = u(t)_{i,j} + \Delta t \cdot \frac{\partial u}{\partial t},\\
v(t+1)_{i,j} = v(t)_{i,j} + \Delta t \cdot \frac{\partial v}{\partial t}
\end{gathered}
\end{equation}
mit

\begin{equation}
\begin{gathered}
\frac{\partial u}{\partial t}_{i,j} = D \cdot \Sigma(t)_{i,j} + \frac{1}{\epsilon} (1-u(t)_{i,j}) \left(u(t)_{i,j}-\frac{v(t)_{i,j}+b}{a}\right)\\
\frac{\partial v}{\partial t}_{i,j} = u(t)_{i,j}^3-v(t)_{i,j}
\end{gathered}
\end{equation}

ausreichend genau. Im Folgenden wird zudem, in Analogie zu \citep{berg2011synchronization}, die Diffusionskonstante auf $D = 1/25$, die Gitterkonstante auf $\Delta x = 0.1$ und die Zeitkonstante auf $\Delta t = 0.01$ gesetzt. Die raumzeitliche Dynamik des Systems ist in Form der $u$-Variable im Anhang in Abbildung \ref{fig:apx_barkley_evolution} dargestellt.. 