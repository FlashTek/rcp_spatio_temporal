\chapter{Fazit}
In dieser Arbeit ist die Anwendung von \textit{Echo State Networks} für die Vorhersage von raumzeitlichen Dynamiken erregbarer Systeme untersucht worden. Dies ist anhand des \textit{Barkley}-, des \textit{Mitchell-Schaeffer}- und des \textit{Bueno-Orovio-Cherry-Fenton}-Modells durchgeführt worden. Für die Vorhersage der raumzeitlichen Dynamiken ist ein sogenannter \textit{Messsondenansatz} entwickelt und verwendet worden, bei dem nur lokal benachbarte Informationen für die Vorhersage genutzt werden.\\

Um die experimentellen Ergebnisse des \textsc{ESN}s einordnen zu können, sind sie mit den Ergebnissen eines \textit{Nächsten-Nachbar}-Ansatzes und \textit{radialer Basisfunktionen} verglichen worden. Dabei sind verschiedene Anwendungsarten betrachtet worden, welche für die wissenschaftliche Untersuchung von Herzen relevant sind: Zuerst ist eine nicht gemessene aus einer gemessenen Systemvariable (Kreuzvorhersage) bestimmt worden. Darauffolgend ist die elektrische Erregung auf der Herzoberfläche anhand des Fernfelds einer Elektrodenmessung rekonstruiert worden. Abschließend ist versucht worden, die elektrische Erregung für unbekannte Regionen aus der Kenntnis weniger Randwerte vorzunehmen.\\

Insgesamt ist deutlich geworden, dass durch die Verwendung der \textit{Echo State Networks} in jedem Anwendungsbereich eine größere Genauigkeit erzielt werden konnte. Daraus folgt, dass die \textsc{ESN}s eine gut geeignete Methode zur Untersuchung und Vorhersage raumzeitlicher Dynamiken erregbarer Medien darstellen. Dieses Ergebnis stimmt mit den aktuellen wissenschaftlichen Erkenntnissen in diesem Bereich überein \citep{Lu2017}.\\

Im Detail konnte die Kreuzvorhersage zwischen den zwei Systemvariablen nahezu fehlerfrei gelöst werden. Dies ist neben dem \textit{Barkley}- und dem \textit{Mitchell-Schaeffer}-Modell auch für \textit{BOCF}-Modell, welches über mehr Systemvariablen verfügt, gelungen. Es bietet sich in weiterführenden Arbeiten an, die Kreuzvorhersage für weitere Systeme mit mehreren Variablen zu betrachten, um den Einfluss der Anzahl der Variablen auf die Genauigkeit der Vorhersage zu quantifizieren. Bei der Rekonstruktion der elektrischen Erregung konnte zudem die makroskopische Struktur der Dynamik rekonstruiert werden, aber nicht die Detailstruktur. Die Vorhersage des unbekannten Bereiches durch die Randwerte konnte nicht zufriedenstellend gelöst werden. Bereits für geringe Größen des vorherzusagenden Bereichs stieg der Fehler sowohl bei dem \textsc{ESN} als auch bei den Referenzmethoden sehr stark an. Dies kann ein Hinweis auf eine generelle physikalische Beschränktheit solcher Voraussagen sein. Es empfiehlt sich auch dieses Phänomen weiter zu untersuchen. \\

Es hat sich zudem herausgestellt, dass die Gewichtsmatrix $\mathbf{W_{out}}$ für verschiedene Bildpunkte eine große Ähnlichkeit aufzeigt. Somit bietet es sich an dieses Verhalten weiter zu betrachten und zu analysieren. Durch die Verwendung einer einzigen Auslesematrix lässt sich die Geschwindigkeit des \textsc{ESN} stark erhöhen. Zusätzlich wäre auch ein Einsatz anderer Methoden aus dem Bereich des \textit{Machine Learnings} als Ersatz für die Auslesematrix, wie beispielsweise \textsc{FFNN}s, denkbar.  