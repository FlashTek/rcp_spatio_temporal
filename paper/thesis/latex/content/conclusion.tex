\chapter{Fazit}
In dieser Arbeit ist die Anwendung von \textit{Echo State Networks} für die Vorhersage von raumzeitlichen Dynamiken erregbarer Systeme untersucht worden. Dies ist anhand des \textit{Barkley}- und des \textit{Mitchell-Schaeffer}-Modells durchgeführt worden. Für die Vorhersage dieser raumzeitlichen Dynamiken ist ein sogenannter \textit{Messsonendenansatz} entwickelt und verwendet worden, bei dem nur lokal benachbarte Informationen für die Vorhersage genutzt werden.\\

Um die experimentellen Ergebnisse des \textsc{ESN}s einzuordnen, sind sie mit den Ergebnissen eines \textit{nächsten Nachbar}-Ansatzes und \textit{radialer Basisfunktionen} verglichen worden. Dabei sind Anwendungsarten betrachtet worden, welche für die Untersuchung von Herzen relevant sind: Zuerst ist eine nicht gemessene aus einer gemessenen Systemvariable bestimmt worden. Anschließend ist die elektrische Erregung auf der Herzoberfläche anhand des Fernfelds einer Elektrodenmessung rekonstruiert worden. Abschließend ist versucht worden die elektrische Erregung für unbekannte Regionen aus der Kenntnis weniger Randwerte vorzunehmen.\\

Es halt sich allgemein gezeigt, dass durch die Verwendung der \textit{Echo State Networks} in jedem Anwendungsbereich eine größere Genauigkeit erzielt werden konnte. Die \textsc{ESN}s stellen eine gut geeignete Methode zur Untersuchung und Vorhersage raumzeitlicher Dynamiken erregbarer Medien dar. Diese Erkenntnis passt zu den neusten wissenschaftlichen Publikationen auf dem Bereich \citep{Lu2017}.\\

Im Detail konnte die Kreuz-Prädiktion zwischen den zwei Systemvariablen nahezu fehlerfrei gelöst werden. Die Rekonstruktion der elektrischen Erregung konnte zudem die makroskopische Struktur der Dynamik rekonstruiert werden, aber nicht die Detailstruktur. Die Vorhersage des unbekannten Bereiches durch die Randwerte konnte nicht zufriedenstellend gelöst werden. Bereits für geringe Größen des vorherzusagenden Bereichs stieg der Fehler sowohl bei dem \textsc{ESN} als auch bei den Referenzmethoden sehr stark an. Dies kann ein Hinweis auf eine generelle physikalische Beschränktheit solcher Voraussagen sein. Es empfiehlt sich dieses Phänomen weitergehend zu untersuchen. \\

Es hat sich zudem herausgestellt, dass die Gewichtsmatrix $\mathbf{W_{out}}$ für verschiedene Bildpunkte eine große Ähnlichkeit aufzeigt. Somit bietet es sich an dieses Verhalten weiter zu untersuchen. Durch die Verwendung einer einzigen Auslesematrix lässt sich die Geschwindigkeit des \textsc{ESN} stark erhöhen. Zusätzlich wäre auch ein Einsatz anderer Methode aus dem Bereich des \textit{Machine Learnings} als Ersatz für die Auslesematrix, wie beispielsweise \textsc{FFNN}s denkbar.  