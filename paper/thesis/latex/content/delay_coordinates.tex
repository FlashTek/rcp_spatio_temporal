\section{Delay Reconstruction}
Die \textit{Delay Reconstructions} (Verzögerung-Konstruktionen) können benutzt werden um Zeitreihen zu analysieren und den Phasenraum des ursprünglichen Systems zu rekonstruieren.
Hierbei wird ein Signal $s(t)$ an diskreten Zeitpunkten betrachtet, sodass sich das diskrete Signal $s_n = s(n\Delta t)$ ergibt. Eine \textit{Delay Reconstruction} erzeugt hier raus ein Signal, in welchem die Informationen $m$ vorheriger Zeitpunkte mit dem Abstand $\tau$ enthalten sind. Somit wird eine höhere dimensionale Zeitreihe $\vec{z}_n \in \mathbf{R}^{m}$ durch
\begin{align}
	\vec{z}_n = \left(s_{n-(m-1)\tau}, s_{n-(m-2)\tau}, \ldots ,s_n \right)
\end{align} 
konstruiert. Bei einer ausreichend hohen Wahl der Rekonstruktionsimension $m$ ist es hiermit möglich den Phasenraum des Attraktors zu rekonstruieren. Für die Wahl der Verzögerungszeit $\tau$ gibt es keine rigorose mathematische Definition oder Beschreibung, sondern es existieren verschiedene Ansätze zur Ermittlung des optimalen Wertes. Ein populärer Ansatz, welcher in dieser Arbeit verwendet besteht darin $\tau$ durch das Auffinden der ersten Nullstelle von der Autokorrelationsfunktion 
\begin{align}
AUC(\tau) = \sum_l^{N-\tau} (s_l-\bar{s})(s_{l+\tau}-\bar{s})
\end{align}   
zu ermitteln. Dies lässt sich dadurch motivieren, dass durch das hinzunehmen des Signals der Zeitreihe die um diesen Wert von $\tau$ verschoben ist, am meisten neue Information hinzugefügt wird, da die Selbstähnlichkeit des Signals am geringsten ist \citep[30\,ff.]{kantz2004nonlinear}