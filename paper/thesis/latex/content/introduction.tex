\chapter{Einführung}
Im Zuge der fortschreitenden technologischen Weiterentwicklung wurden Wissenschaftler mit neuen Methoden ausgestattet, um Herzen zu untersuchen. So ist es möglich, bei sogenannten \textit{in vitro} Experimenten, ein Herz außerhalb des eigentlichen Körpers schlagen zu lassen, um sein Verhalten studieren zu können. Dabei wird die Dynamik des Herzmuskels durch unterschiedliche Ionenarten beeinflusst. Diese erzeugen sowohl im Inneren, als auch auf der Oberfläche des Herzens ein elektrisches Potential, womit eine Spannung messbar ist. Mit den aktuellen Techniken ist es möglich, bei solchen Experimenten, die mechanische Kontraktion des Herzen und auch den Spannungsverlauf auf der Oberfläche zu messen. Doch Informationen über die Ströme und Spannungen im Inneren des Herzens sind aufgrund der erschwerten Zugänglichkeit kaum messbar. Ebenso können kaum Ströme einzelner bestimmter Ionenarten vermessen werden. \\

Im Rahmen dieser Arbeit wird versucht eine Möglichkeit zu finden, um sowohl die im Inneren auftretenden Spannungen, als auch weitere schwer messbare (verborgene) Systemvariablen zu approximieren. Dabei werden Methoden des \textit{Machine Learning} (maschinelles Lernen) verwendet. Hierfür wird das \textit{Reservoir Computing} anhand der \textit{Echo State Networks} angewendet und anschließend mit klassischen, bereits länger bestehenden, Methoden verglichen. Dies sind die Approximation mittels \textit{Nächster-Nachbarn} und \textit{radialer Basisfunktionen}.\\
  
Weil die Anwendung solcher Techniken auf echte Messdaten mit einigen Hindernissen verbunden ist, werden zuerst drei Modellsysteme betrachtet. Diese Modelle sind in der Vergangenheit entwickelt worden, um eine mathematische Beschreibung der Funktionsweise der Herzen zu geben und ihre Dynamiken zu beschreiben. In dieser Arbeit werden das \textit{Barkley}-, das \textit{Mitchell-Schaeffer} und das \textit{Bueno-Orovio-Cherry-Fenton}-Modell untersucht. Die Modelle gehören zu der Klasse der \textit{erregbaren Medien} und zeigen auch raumzeitliches Chaos.\\

Zu Beginn wird in Kapitel \ref{ch:theory} ein theoretischer Überblick gegeben. Dort werden zunächst die drei verwendeten Modelle eingeführt und beschrieben, und im Anschluss die beiden klassischen Vorhersagemethoden vorgestellt. Darauffolgend wird in Kapitel \ref{sc:esn} die Technik der \textit{Echo State Networks} eingeführt und beschrieben. In Kapitel \ref{ch:experiments} werden diese Ansätze nun an drei verschiedenen Szenarien getestet für das \textit{Barkley}- und das \textit{Mitchell-Schaeffer}-Modell und verglichen: Zuerst wird in Kapitel \ref{sec:exp_cross_pred} eine verborgene Variable auf Grundlage einer gemessenen Variable approximiert (Kreuzvorhersage). Dies wird ebenfalls für das \textit{Bueno-Orovio-Cherry-Fenton}-Modell durchgeführt. Danach werden die Ansätze genutzt um die Qualität der gemessenen Spannungsverläufe in \ref{sec:exp_unblur} zu erhöhen. Anschließend wird in Abschnitt \ref{sec:exp_cross_pred} die Spannung im Inneren eines solchen Systems vorhergesagt.\\

Die einzelnen Ansätze sind in der Programmiersprache \textsc{Python 3.5.2} implementiert. Zusätzlich werden die Bibliotheken \textsc{numpy} und \textsc{scipy} zur Berechnung genutzt. Mithilfe der Bibliothek \textsc{matplotlib} werden die graphischen Darstellungen erzeugt. Der somit angefertigte und dokumentierte Quellcode ist auf \textsc{GitHub} unter der Adresse \href{https://github.com/FlashTek/rcp\_spatio\_temporal}{https://github.com/FlashTek/rcp\_spatio\_temporal} einzusehen. Der schriftlichen Ausführung ist er zudem auf einer CD beigelegt.