\section{Prädiktion der Dynamik durch das Fernfeld}
Bei der Durchführung von invitro Experimenten mit Herzen gibt es verschiedene Möglichkeiten die Messung der elektrischen Erregung auf der Herzoberfläche durchzuführen. Zum einen können Elektroden zur Messung benutzt werden, zum anderen allerdings auch Fluoreszenzmessungen durchgeführt werden. Bei der Verwendung von Elektroden wird effektiv nicht das unmittelbare elektrische Feld auf der Herzoberfläche gemessen, sondern ein Fernfeld dessen. Es stellt sich nun die Frage, ob aus der Kenntnis dieses Fernfeldes die korrekte Erregung auf der Oberfläche bestimmt werden kann. Eine experimentelle Untersuchung dieser Fragestellung wird im Folgenden durchgeführt. Hierfür müssen zuerst diese Fernfeldaufnahmen für das \textit{Barkley}- und für das \textit{Mitchell-Schaeffer}-Modell erzeugt werden. Dabei wird das Fernfeld nicht korrekt simuliert, sondern durch eine gaußsche Unschärfe emuliert. Dazu wird auf das gesamte Feld der Spannungsvariable beider Modelle eine solche Unschärfe mit einer Breite $\sigma_{Blur} = 8.0$ mittels einer Faltung angewendet. 